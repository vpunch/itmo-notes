\documentclass[a4paper]{article}
\usepackage[utf8]{inputenc}
\usepackage[T2A]{fontenc}
\usepackage[english,russian]{babel}

\usepackage[left=40mm,right=40mm,top=25mm,bottom=30mm]{geometry}
\usepackage[skip=.8\baselineskip]{parskip}

\usepackage{setspace}
\setstretch{1.10}

\usepackage{enumitem}
\setlist{
    labelsep=0pt,
    labelwidth=1.4em,
    leftmargin=\labelwidth,
    align=left
}


\usepackage{mathtools}
\usepackage{amssymb}
\usepackage{amsthm}
\usepackage{cancel}
\usepackage{bm}
% символы для теоретической информатики (например [[)
\usepackage{stmaryrd}

\newcommand{\pluseq}{\mathrel{+}=}
\newcommand{\pade}[2][]{\frac{\partial #1}{\partial #2}}


\DeclareMathOperator*{\argmin}{argmin}
\DeclareMathOperator*{\argmax}{argmax}
\DeclareMathOperator{\sgn}{sign}
% диагональная матрица
\DeclareMathOperator{\diag}{diag}


\usepackage{float}
\newcommand{\fig}[2][1]{
    \begin{figure}[H]
        \centering
        \includegraphics[width=#1\textwidth]{figs/#2}
    \end{figure}
}

\newenvironment{tbl}[1]{
    \begin{table}[h]
        \small
        \centering
        \begin{tabular}{#1}
        }{
        \end{tabular}
    \end{table}
}

\usepackage{array}
\newcolumntype{R}[1]{>{\raggedleft\arraybackslash}p{#1}}
\newcolumntype{L}[1]{>{\raggedright\arraybackslash}p{#1}}


\usepackage{url}
\usepackage{hyperref}
