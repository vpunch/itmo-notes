% This work is licensed under the Creative Commons Attribution-NonCommercial
% 4.0 International License. To view a copy of this license, visit
% http://creativecommons.org/licenses/by-nc/4.0/ or send a letter to Creative
% Commons, PO Box 1866, Mountain View, CA 94042, USA.

\documentclass[a4paper]{article}
\usepackage[utf8]{inputenc}
\usepackage[T2A]{fontenc}
\usepackage[english,russian]{babel}

\usepackage[left=40mm,right=40mm,top=25mm,bottom=30mm]{geometry}
\usepackage[skip=.8\baselineskip]{parskip}

\usepackage{setspace}
\setstretch{1.10}

\usepackage{enumitem}
\setlist{
    labelsep=0pt,
    labelwidth=1.4em,
    leftmargin=\labelwidth,
    align=left
}


\usepackage{mathtools}
\usepackage{amssymb}
\usepackage{amsthm}
\usepackage{cancel}
\usepackage{bm}
% символы для теоретической информатики (например [[)
\usepackage{stmaryrd}


\DeclareMathOperator*{\argmin}{argmin}
\DeclareMathOperator*{\argmax}{argmax}
\DeclareMathOperator{\sgn}{sign}


\usepackage{float}
\newcommand{\fig}[2][1]{
    \begin{figure}[H]
        \centering
        \includegraphics[width=#1\textwidth]{figs/#2}
    \end{figure}
}

\newenvironment{tbl}[1]{
    \begin{table}[h]
        \small
        \centering
        \begin{tabular}{#1}
        }{
        \end{tabular}
    \end{table}
}

\usepackage{array}
\newcolumntype{R}[1]{>{\raggedleft\arraybackslash}p{#1}}
\newcolumntype{L}[1]{>{\raggedright\arraybackslash}p{#1}}


\usepackage{url}
\usepackage{hyperref}



\begin{document}

\section{Установочная лекция (водная)}

\textbf{Трансляция} --- использование математики и коммуникационных технологий
для установления связей межну разными предметными областями и перехода от науки
к практике.

Продукт трансляции --- \textbf{цифровой образ}. Образ --- это модель объекта.

Знания транслируются, процессы трансформируются.

Ключевой механизм трансляции (обеспечения трансформации) --- получение новых
знаний за счет цифрового воспроизведения процессов реального мира для их
исследования.

Методология --- раздумие о том, как можно достигать довых знаний через цифровое
воспроизведение объектов предметной области.

Цикл трансляционных исследований:
\begin{itemize}
    \item Анализ предметной области, обоснование гипотез, выбор путей их
        исследования

    \item Формализация задачи и создание описательной модели

    \item Разработка математического аппарата для работы с моделью

    \item Насыщение цифрового образа на данных

    \item Разработка путей исследования объекта реального мира на основе
        цифрового образа

    \item Получение и анализ первичных результатов

    \item Разработка методик использования цифрового образа

    \item Анализ пользовательского опыта
\end{itemize}

\section{Современные методы анализа фактографической информации}

\textbf{SciVal} --- аналитический инструмент для оценки и планирования
научно-исследовательской деятельности на основе данных Scopus. Scopus --- база
данных научных статей.

SciVal позволяет следить за своей публикационной активностью, сравнивать себя с
другими учеными, создавать научные группы.

С итмошным электронным адресом можно пользоваться бесплатно.

\textbf{Web of Science} --- аналогичный инструмент анализа научных статей.

\textbf{VOSviewer} --- инструмент анализа ключевых слов в публикациях. Создаем
новый проект, выбираем анализ по библиографическим данным, загружаем данные из
WoS, выбираем тип анализа co-occurrence (соупотребление ключевых слов).
Позволяет выделить направления, которым стоит уделать больше внимания в данной
сфере.

\textbf{Виды обзоров}:
\begin{itemize}
    \item Ретроспективный. Создание представления об актуальных направлениях
        исследований и state-of-the-art результатах. От 10 страниц, от 15
        источников.

    \item Сравнительный. Обоснование новизны и актуальности научного
        исследования, поиск сходств и различий с более ранними методами. От 1
        страницы, от 10 источников.

    \item Критический. Анализ истории развития и текущего состояния научной
        облатси, критическая оценка и систематизация исследований в области. От
        30 страниц, от 50 источников.
\end{itemize}

\textbf{Google Scholar} --- поисковик по статьям.

Related Work --- раздел в статьях, который содержит краткое описание более
ранних или конкурирующих методов.

Рейтинг конференций можно посмотреть тут http://portal.core.edu.au/conf-ranks/.
Аналогичный инструмент для проверки статуса журнала https://www.scimagojr.com.

Для цитирования в Google Scholar есть очень удобная кнопка под ссылкой.

Выбирать источники не старше 5 лет. Хороший тон приводить шаги выполнения
обзора (воспроизводимость научных исследований). Хорошая практика --- сравнение
алгоритмов или решений по единой системе криериев (таблица).

Классификатор РНФ (Российский Научный Фонд):
\url{https://rscf.ru/contests/classification/}.

\section{Системный анализ и информационное моделирование}

Системный анализ позволяет описать систему формально как совокупность
компонентов и связей между ними.

\textbf{Объект} --- реальный предмет, который необходимо изучить.

\textbf{Система} --- объект, который представляется в виде совокупности
элементов, связанных некоторыми отношениями, которые образуют целостность
(\textbf{эмерджентность}). Части системы можно называть \textbf{подсистемами}.

\textbf{Инвариантность} --- свойство неизменности при переходе из одной системы
в другую.

\textbf{Виды целей}:
\begin{itemize}
    \item Глобальная (goal). Общая цель, которая достигается
        согласованной деятельностью всех компонент системы.

    \item Промежуточная (objective). Цель для конкретного шага деятельности
        системы.

    \item Частная (sub-goal). Цель деятельности конкретной компоненты системы
        (может не совпадать с глобальной целью).
\end{itemize}

\textbf{Задача} (task, problem) --- множество исходных посылок (входные данные).

\textbf{Синергетический эффект} --- взаимодействие входящих в систему элементов,
приводящее к более высокой результативности, чем сумма результативностей
элементов по отдельности.

\textbf{Внутренняя среда} --- среда, в которой находятся компоненты системы и которая
определяет их функционирование.

\textbf{Внешняя среда} --- среда, в которой нет элементов системы, с которой система
может взаимодействовать и осуществлять ресурсный обмен.

\textbf{Тектологическая граница} --- область соприкосновения взаимодействия системы с
внешней средой.

\textbf{Системный анализ} --- методология исследования сложных проблем теории и
практики.

\textbf{Описание системы} --- формализация всех ее элементов, их взаимосвязей,
функционирования и определение множества всех состояний.

Если исследуемую систему не удается описать с помощью известных терминов,
понятий и обозначений, то такая система является \textbf{плохо формализуемой}.

Системе можно дать внутреннее и внешнее описание. В первом случае мы описываем
внутреннюю структуру системы. Во втором случае мы рассматриваем систему как
черный ящик (ничего не знам о том, что внутри), изучаем поведение по отклкику
при различных воздействиях на систему. Так можно заменить внутреннее описание
системы линейной регрессией, если она достаточно хорошо аппроксимирует
поведение.

Базовые топологии структуры: линейная, иерархическая, матричная, сетевая
(связи выстраиваются произвольно).

Если информации для описания структуры недостаточно или структуру описать
нельзя, то такая система называется слабоструктурированной.

Если возможен обмен между любыми двумя подсистемами, то структура связная.

Система называется сложной если ее структура не является линейной , подсистемы
находятся на разных уровнях (связи возможны между элементами на одном уровне и
на разных уровнях), число элементов на уровне невозможно охватить.

Масштабы системы (уровни) могут быть атомный, молекулярный, клеточный, тканей,
органов и т.д.

Виды описания систем:
\begin{itemize}
    \item Морфологическое (структурное или топологическое). Описание структуры,
        совокупоности элементов

    \item Функциональное. Законы функционирования, алгоритмы поведения (работы)
        системы (например, IDEF0).

    \item Информационное. Описание информационных связей, окружающей среды.
\end{itemize}

Некоторые методы системного анализац
\begin{itemize}
    \item Абстрагирование и конкретизация

    \item Анализ и синтез (индукция и дедукция)

    \item Композиция и декомпозиация

    \item Линеаризация и выделение нелинейных составляющих

    \item Макетирование

    \item Ресурсный анализ

    \item Верификация

    \item Кластеризация и классификация

    \item Распознавание
\end{itemize}

Виды ресурсов в ресурсном анализе:
\begin{itemize}
    \item вещество (таблица Менделеева)
    \item энергия (изменчивость материи),
    \item информация (мера структурированности материи)
    \item человек (ресурс общества)
    \item организованность (упорядоченность ресурсов)
    \item простанство (мера протяженности материи)
    \item время (мера обратимости материи, событий)
\end{itemize}

Метод Оптнера.

Информация --- интерпретация данных.

\section{Непрерывные математические модели}

\textbf{Типы моделей}:
\begin{itemize}
    \item Эвристические модели (образы, возникают в воображении человека)

    \item Натурные модели --- подобны реальным системам, отличие состоит в
        размерах, числе и метериале элементов.

    \item Математические модели (выполняются на математическом языке, формулы)

    \item Промежуточные виды (например, графические).
\end{itemize}

\textbf{Задачи моделирования}:
\begin{itemize}
    \item Прямая. Стурктура модели и все ее параметры считаются известными.
        Нужно провесит исследование модели для извлечения полезного знания об
        объекте

    \item Обратная. Известно множество моделей, нужно выборать одну на
        основании дополнительных данных об объекте (задача оценки параметров).
\end{itemize}

\textbf{Дискретная модель} рассматриваем объекты как дискретные (молекулы, состояния).
\textbf{Непрерываная модель} непрерывно представляет объекты (температура, скорость).

\textbf{Детерминированная модель} всегда работает одинаково для заданного набора
начальных условий.

Пример непрерывной дискретной математической модели является логистическая
функция $y(t) = \frac{1}{1 + e^{-t}}$. Также называется \textbf{моделью роста}.
Другой пример --- линейная функция.

\textbf{Модель Мальтуса} описывает рость населения. Он является экспоненциальным: $P(t)
= P_0 e^{rt}$. Где $P_0$ --- исходная численность населения, $r$ --- темп
прироста неселения, $t$ --- время.

Модель Ферхюльста учитывает естественную границу населения (емкость среды), которая
ограничивает рост: $\frac{dP}{dt} = rP(1 - \frac{P}{K})$, где $K$ --- емкость
среды. Здесь скобки за скобками общий множитель, а не функция.


Уравнения системной динамики (stock and flow, source and sink). Составляющие:
\begin{itemize}
    \item 
\end{itemize}

У непрерывных моделей есть ряд ограничений.

У нас должно быть большое количество объектов и они должны быть однородными.
Например, толпу людей можно рассматривать как поток. Иначе интепретация работы
модели будет нереалистичной.

\section{Технологии использования результатов в предметных областях}

Трансляционные исследования (ТИ) предназначены для нахождения применений
результатов фундаметальных исследований.

\end{document}
